%%%%%%%%%%%%%%%%%%%%%%%%%%%%%%%%%%%%%%%%%
% Beamer Presentation
% LaTeX Template
% Version 1.0 (10/11/12)
%
% This template has been downloaded from:
% http://www.LaTeXTemplates.com
%
% License:
% CC BY-NC-SA 3.0 (http://creativecommons.org/licenses/by-nc-sa/3.0/)
%
%%%%%%%%%%%%%%%%%%%%%%%%%%%%%%%%%%%%%%%%%

%----------------------------------------------------------------------------------------
%	PACKAGES AND THEMES
%----------------------------------------------------------------------------------------

\documentclass{beamer}
\usepackage{CJK}
\usepackage[normalem]{ulem}

\mode<presentation> {

% The Beamer class comes with a number of default slide themes
% which change the colors and layouts of slides. Below this is a list
% of all the themes, uncomment each in turn to see what they look like.

%\usetheme{default}
%\usetheme{AnnArbor}
%\usetheme{Antibes}
%\usetheme{Bergen}
%\usetheme{Berkeley}
%\usetheme{Berlin}
%\usetheme{Boadilla}
%\usetheme{CambridgeUS}
%\usetheme{Copenhagen}
%\usetheme{Darmstadt}
%\usetheme{Dresden}
%\usetheme{Frankfurt}
%\usetheme{Goettingen}
%\usetheme{Hannover}
%\usetheme{Ilmenau}
%\usetheme{JuanLesPins}
%\usetheme{Luebeck}
\usetheme{Madrid}
%\usetheme{Malmoe}
%\usetheme{Marburg}
%\usetheme{Montpellier}
%\usetheme{PaloAlto}
%\usetheme{Pittsburgh}
%\usetheme{Rochester}
%\usetheme{Singapore}
%\usetheme{Szeged}
%\usetheme{Warsaw}

% As well as themes, the Beamer class has a number of color themes
% for any slide theme. Uncomment each of these in turn to see how it
% changes the colors of your current slide theme.

%\usecolortheme{albatross}
%\usecolortheme{beaver}
%\usecolortheme{beetle}
%\usecolortheme{crane}
%\usecolortheme{dolphin}
%\usecolortheme{dove}
%\usecolortheme{fly}
%\usecolortheme{lily}
%\usecolortheme{orchid}
%\usecolortheme{rose}
%\usecolortheme{seagull}
%\usecolortheme{seahorse}
\usecolortheme{whale}
%\usecolortheme{wolverine}

%\setbeamertemplate{footline} % To remove the footer line in all slides uncomment this line
%\setbeamertemplate{footline}[page number] % To replace the footer line in all slides with a simple slide count uncomment this line

%\setbeamertemplate{navigation symbols}{} % To remove the navigation symbols from the bottom of all slides uncomment this line
}

\usepackage{graphicx} % Allows including images
\usepackage{booktabs} % Allows the use of \toprule, \midrule and \bottomrule in tables

%----------------------------------------------------------------------------------------
%	TITLE PAGE
%----------------------------------------------------------------------------------------

\title[SJTUG Mirrors]{\texorpdfstring{\sout{How to make a big news}}{} SJTUG Mirrors}

\author{Lv Zheng}
\institute[SJTUG]
{
SJTUG\\
\medskip
\textit{lv.zheng.2015@gmail.com}
}
\date{April 23, 2016}

\begin{document}

%----------------------------------------------------------------------------------------
%	PRESENTATION SLIDES
%----------------------------------------------------------------------------------------

\begin{frame}[fragile]
\frametitle{Big News}
\begin{CJK}{UTF8}{gbsn}

骆神 wrote in the News Section:\\~

喜迎校庆 SJTUG Mirrors加入Arch Linux镜像源\\~

在上海交通大学120周年校庆前夕,经过SJTUG所有人的共同努力,我们上线了Arch
Linux,PuTTY和Cygwin的镜像,向百廿交大的生日献上我们的祝福。\\~

Arch Linux源使用方法:\\~

编辑/etc/pacman.d/mirrorlist,先注释掉里面的所有行,然后在文件的最顶端添加
\end{CJK}

\begin{verbatim}
Server = http://mirrors.sjtug.org/archlinux/$repo/os/$arch
\end{verbatim}

\end{frame}

%------------------------------------------------

\begin{frame}
\titlepage
\end{frame}

%------------------------------------------------

\begin{frame} 
\frametitle{Questions to Ask}
\begin{block}{Q0}
How great is the server's hardware?
\end{block}
\begin{block}{Q1}
How does a mirror work?
\end{block}
\begin{block}{Q2}
How do we deploy the services?
\end{block}
\end{frame}

%------------------------------------------------

\begin{frame}
\huge{\centerline{How great is the server's hardware?}}
\large{\centerline{Let's have a look!}}
\end{frame}

%------------------------------------------------

\begin{frame}
\huge{\centerline{How does a mirror work?}}
\end{frame}

%------------------------------------------------

\begin{frame}[fragile]
\frametitle{tunasync}

Different websites/repos require different tools to synchronize, and the most
used tool for synchronization is \texttt{rsync}. However, some sites such as
\textit{Debian archive} requires 2-stage sync, while some others don't.\\~

\textbf{\texttt{tunasync}} is a great tool for such operations. It supports
different sync protocols. Sites to sync are configured in a config file. For
example, to sync PuTTY (which is a very small site, less than 100 MB in size),
add these lines to the config file:

\begin{block}{tunasync.conf}
\begin{verbatim}
[[mirrors]]
name = "PuTTY"
provider = "rsync"
upstream = "rsync://rsync.chiark.greenend.org.uk/\
	ftp/users/sgtatham/putty-website-mirror/"
\end{verbatim}
\end{block}

\sout{Too young} too simple, isn't it?

\end{frame}

%------------------------------------------------

\begin{frame}
\frametitle{nginx}
\texttt{nginx} is a even faster HTTP server program than \sout{western} Apache
HTTPD, especially for serving static contents, thus very suitable for our
mirrors. However, dynamic contents such as PHP generated pages can be served
with FastCGI.
\end{frame}

%------------------------------------------------

\begin{frame}
\huge{\centerline{How do we deploy the services?}}
\end{frame}

%------------------------------------------------

\begin{frame}
\frametitle{Traditional way (1)}
Install all the required software packages such as Apache, MariaDB, and PHP on
the host and run them on the host space.\\~

Pros:
\begin{itemize}
\item ......
\end{itemize}

Cons:
\begin{itemize}
\item One security flaw of a software may break the entire system. (may be
avoided by using a non-privileged user for each service)
\item Hard to deploy, especially hard for cluster deployment.
\item Hard to deploy multiple services that use the same software.
\item Hard to make backups.
\end{itemize}

\end{frame}

%------------------------------------------------

\begin{frame}
\frametitle{Traditional way (2)}
Virtually same as the previous way. Just do everything in a virtual machine
instead.\\~

Pros:
\begin{itemize}
\item Easy for cluster deployment and backups.
\item May deploy multiple services on a single computer.
\item Strong isolation.
\end{itemize}

Cons:
\begin{itemize}
\item Performance downgrade. Low memory-efficiency. Cache unfriendly.
\item It's excit\sout{ed}ing to do so on a cloud server :)
\end{itemize}

\end{frame}

%------------------------------------------------

\begin{frame}
\huge{\centerline{The Docker way}}
\end{frame}

%------------------------------------------------

\begin{frame}
\frametitle{What is Docker}
\begin{itemize}
\item Docker allows you to package an application with all of its dependencies
into a standardized unit for software development.
\item Docker containers wrap up a piece of software in a complete filesystem
that contains everything it needs to run: code, runtime, system tools, system
libraries - anything you can install on a server. This guarantees that it will
always run the same, regardless of the environment it is running in.\\~

\begin{CJK}{UTF8}{gbsn}
(凑表脸地偷懒了)
\end{CJK}
\end{itemize}
\end{frame}

%------------------------------------------------

\begin{frame}
\Large{\centerline{Let's learn Docker from an example}}
\end{frame}

%------------------------------------------------

\begin{frame}
\begin{CJK}{UTF8}{gbsn}
\Huge{\centerline{批判一番}}
\end{CJK}
\end{frame}

%------------------------------------------------

\begin{frame}
\frametitle{Rewrite the whole project thoroughly}
Having learned about the \texttt{mirror-docker} project, I decided to rewrite
the \texttt{Dockerfile}s and scripts.\\~

With our joint efforts, the project was finally rewritten and put into
application.\\~

We added \texttt{archlinux}, \texttt{PuTTY}, and \texttt{cygwin} repos right
before SJTU 120 Anniversary.
\end{frame}

%------------------------------------------------

\begin{frame}
\frametitle{Structure of the new settings}
Currently there are two Docker containers running on the server, and another
container running on demand.

\begin{enumerate}
\item \texttt{tunasync} container syncs the repos and generate
\texttt{tunasync.json} regularly. Files are stored in a host directory mounted
to the container.
\item \texttt{nginx} container serves the HTTP content. On each request, it
first looks up the \texttt{site} directory for web content. If not found, serve
the \texttt{tunasync}ed files.
\item \texttt{jekyll} container builds the static web pages from source. It is
manually run on demand, for example, when news are updated.

\end{enumerate}
\end{frame}

%------------------------------------------------

\begin{frame}
\begin{CJK}{UTF8}{bsmi}
\huge{\centerline{識得唔識得啊}}
\large{\centerline{Sik Dak Ng Sik Dak Aa}}
\end{CJK}
\end{frame}

%------------------------------------------------

\begin{frame}
\frametitle{Who is to blame}
\begin{CJK}{UTF8}{gbsn}
\begin{itemize}
\item 骆神 \sout{gives us a negative example of \texttt{Dockerfile},} 钦定 the
entire project, and 骗来服务器。
\item 李臻博 \sout{violenty} upgrades docker on the server and contributes a
lot of precious ideas and script code to the rewrite.
\item 吕正 does \sout{destructive} experiments and deploys the services on the
server.
\item And thank you all for testing the mirrors.
\end{itemize}
\end{CJK}
\end{frame}

%------------------------------------------------

\begin{frame}
\frametitle{What Next?}
0.1 Milestone:
\begin{enumerate}
\item System monitor
\item HTTPS support
\end{enumerate}
1.0 Milestone:
\begin{enumerate}
\item Rewrite the Web frontend
\end{enumerate}
Also,
\begin{enumerate}
\item Establish a working group for mirrors.
\item Find a better way to manage privileges on the server
\item Obtain more computing resources, especially disk space and bandwidth
\end{enumerate}

\end{frame}

%------------------------------------------------

\begin{frame}
\huge{\centerline{The End}}
\tiny{\centerline{Terribly sorry for broken English}}
\end{frame}

%----------------------------------------------------------------------------------------

\end{document} 
